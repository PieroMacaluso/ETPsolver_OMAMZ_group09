\documentclass[a4paper]{beamer}
\usepackage[italian]{babel}
\usepackage[utf8]{inputenc}
\usepackage[T1]{fontenc}

\usepackage{libertine}
\usetheme[progressbar=foot]{metropolis}

\title[The title]{OMA 9}
\subtitle{Abbiamo fatto cose}
\author{Autori vari}
\institute{Politecnico di Torino}
\date{2018}

\definecolor{ilcolore}{HTML}{000099}

%\setbeamercolor{title}{fg=ilcolore}
%\setbeamercolor{frametitle}{fg=ilcolore}
%\setbeamercolor{structure}{fg=ilcolore}
%\setbeamercolor{normal text}{ fg=black , bg=white }
\setbeamercolor{alerted text}{fg=ilcolore}
\setbeamercolor{example text}{fg=ilcolore}
\setbeamercolor{progress bar}{fg=ilcolore}
\setbeamercolor{frametitle}{fg=white, bg=ilcolore}
%\setbeamercolor{background canvas}{fg=ilcolore}

\begin{document}
	
	\frame{\titlepage}
	
	\begin{frame}
		\frametitle{Table of Contents}
		\tableofcontents
	\end{frame}
	
	\section{Le cose che abbiamo provato}
	
	\begin{frame}
		\frametitle{Tentativi ed errori}
		Sono tanti e var\^i:
		\begin{itemize}
			\item Digitare codice con le dita
			\item Dare testate al muro
			\item Piangere
		\end{itemize}
	\end{frame}

	\begin{frame}
		\frametitle{In sintesi}
		Il SA è bello
	\end{frame}

	\section{Le cose che hanno funzionato}
	
	\begin{frame}
		\frametitle{Things that worked}
		Il SA
	\end{frame}

	\section{Conclusioni}
	
	\begin{frame}
		\frametitle{Important findings}
		SA is g\"ud.
	\end{frame}
\end{document}