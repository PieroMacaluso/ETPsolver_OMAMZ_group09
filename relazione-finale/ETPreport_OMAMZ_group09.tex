\documentclass[11pt, a4paper, leqno]{article}
\usepackage[italian]{babel}
\usepackage[utf8]{inputenc}
\usepackage[T1]{fontenc}
\usepackage{parskip}
\usepackage{libertine} % font bello
\usepackage[paper=a4paper,top=1in,bottom=1.1in,right=1in,left=1in]{geometry} % margini

\usepackage{enumitem} % liste più compatte (se ce ne saranno)
\setlist[enumerate]{itemsep=0.0em}
\setlist[itemize]{itemsep=0.0em}
\usepackage[compact]{titlesec}
\usepackage{amsmath}
\usepackage{amssymb}
\usepackage[]{algorithm2e}

\newcommand{\nome}[2]{
\begin{minipage}[t]{0.185\linewidth}
	\centering #1\par
	\centering\small (#2)\par
\end{minipage}
}

% Questo era utile per evidenziare le cose "dubbie e da correggere"
% Poi si elimina il comando cosi' non restano \boh{} in giro
%\usepackage{xcolor}
%\newcommand{\boh}[1]{\textcolor{red}{#1}}

\newcommand{\eqp}[1]{\textit{(eq. \ref{#1})}}
\newcommand{\eq}[1]{\textit{\ref{#1}}}

\begin{document}	
	\begin{center}
		{\huge\textbf{OMA Final Report}}\par
		\vspace{0.3em}
		{\large\textbf{Group 9}}\par
		\vspace{1em}
		\nome{Piero Macaluso}{s252894}
		\nome{Lorenzo Manicone}{s217189}
		\nome{Donato Tortoriello}{s205639}
		\nome{Ludovico Pavesi}{s246422}
		\nome{Alberto Romano}{s254036} % 164371
	\end{center}

	\section{Initial solution generation}
	
	\subsection{Random}
	
	Initially we tried generating random starting solutions: that is, to each exam assign a random time slot. This was very fast, but even after thousands of runs we couldn't find even a single feasible solution on ``real'' instances. It worked on the ``test'' instance, however.
	
	\subsection{Greedy 1 (``Extremely Stupid Greedy'')}
	
	We tried to implement a very simplistic greedy algorithm:
	
	\begin{algorithm}[H]
		%\KwData{dati}
		%\KwResult{risultato}
		%initialization\;
		\ForEach{exam}{
			\ForEach{timeslot}{
				\If{exam can be scheduled in timeslot}{
					assign exam to timeslot\;
					break (evaluate next exam)\;
				}
			}
		}
		\caption{The very simplistic greedy algorithm}
	\end{algorithm}

	This never yielded a feasible solution, as when it got near the end of the exam list there were no more non-conflicting time slots available.
	
	It should be noted that this algorithm is deterministic, as it always produced the same solution.
	
	\subsection{Greedy 2 (``Slightly More Cultured Greedy'')}
	
	An attempt was made to introduce nondeterminism in the previously mentioned greedy algorithm. This was achieved by randomizing the order in which time slots are examined: again, this never yielded a single feasible solution, even after hundreds of runs.
	
	\subsection{FFS Algorithm}
	
	\dots
	
	\section{Further solutions generation}
	
	\subsection{Genetic algorithm}
	
	We tried running a genetic algorithm implementation anyway, with ``Random'' and ``Greedy 2'' initial solutions especially. It worked fine in the test instance, reaching the optimum solution in around 5-10 iterations with single-point crossover and a mutation (assigning a random time slot to an exam) probability of $0.1$, on real instances it never produced a feasible solution, not even after thousands of iterations.
	
	\subsection{Simulated mutant genetic annealing of doom\texttrademark}
	
	\dots
	
\end{document}
