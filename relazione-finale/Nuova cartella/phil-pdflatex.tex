%%%%%%%%%%%%%%%%%%%%%%%%%%%%%%%%%%%%%%%%%%%%%%%%%%%%%%%%%%%%%%%%%%%%
%% I, the copyright holder of this work, release this work into the
%% public domain. This applies worldwide. In some countries this may
%% not be legally possible; if so: I grant anyone the right to use
%% this work for any purpose, without any conditions, unless such
%% conditions are required by law.
%%%%%%%%%%%%%%%%%%%%%%%%%%%%%%%%%%%%%%%%%%%%%%%%%%%%%%%%%%%%%%%%%%%%

\documentclass{beamer}
\usetheme[faculty=phil]{fibeamer}
\usepackage[utf8]{inputenc}
\usepackage[
  main=english, %% By using `czech` or `slovak` as the main locale
                %% instead of `english`, you can typeset the
                %% presentation in either Czech or Slovak,
                %% respectively.
  czech, slovak %% The additional keys allow foreign texts to be
]{babel}        %% typeset as follows:
%%
%%   \begin{otherlanguage}{czech}   ... \end{otherlanguage}
%%   \begin{otherlanguage}{slovak}  ... \end{otherlanguage}
%%
%% These macros specify information about the presentation
\title{Examination Timetabling} %% that will be typeset on the
\subtitle{Optimization Methods and Algorithms Group 9} %% title page.
\author{s252894 - Piero Macaluso\\s246422 - Ludovico Pavesi\\s254036 - Alberto Romano\\s217189 - Lorenzo Manicone\\s246422 - Donato Tortoriello}
%% These additional packages are used within the document:
\usepackage{ragged2e}  % `\justifying` text
\usepackage{booktabs}  % Tables
\usepackage{tabularx}
\usepackage{tikz}      % Diagrams
\usetikzlibrary{calc, shapes, backgrounds}
\usepackage{amsmath, amssymb}
\usepackage{url}       % `\url`s
\usepackage{listings}  % Code listings
\frenchspacing
\begin{document}
  \frame{\maketitle}

  \AtBeginSection[]{% Print an outline at the beginning of sections
    \begin{frame}<beamer>
      \frametitle{Outline for Section \thesection}
      \tableofcontents[currentsection]
    \end{frame}}

   \section{Le cose che abbiamo provato}
   
   \begin{frame}
   \frametitle{Tentativi ed errori}
   Sono tanti e var\^i:
   \begin{itemize}
   	\item Digitare codice con le dita
   	\item Dare testate al muro
   	\item Piangere
   \end{itemize}
\end{frame}

\begin{frame}
\frametitle{In sintesi}
Il SA è bello
\end{frame}

\section{Le cose che hanno funzionato}

\begin{frame}
\frametitle{Things that worked}
Il SA
\end{frame}

\section{Conclusioni}

\begin{frame}
\frametitle{Important findings}
SA is g\"ud.
\end{frame}
\end{document}
