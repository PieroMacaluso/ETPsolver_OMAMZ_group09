%%%%%%%%%%%%%%%%%%%%%%%%%%%%%%%%%%%%%%%%%%%%%%%%%%%%%%%%%%%%%%%%%%%%
%% I, the copyright holder of this work, release this work into the
%% public domain. This applies worldwide. In some countries this may
%% not be legally possible; if so: I grant anyone the right to use
%% this work for any purpose, without any conditions, unless such
%% conditions are required by law.
%%%%%%%%%%%%%%%%%%%%%%%%%%%%%%%%%%%%%%%%%%%%%%%%%%%%%%%%%%%%%%%%%%%%

\documentclass[notes]{beamer}       % print frame + notes
%\documentclass[notes=only]{beamer}   % only notes
%\documentclass{beamer} 
\usetheme[faculty=phil]{fibeamer}
\usepackage[utf8]{inputenc}
\usepackage[
  main=english, %% By using `czech` or `slovak` as the main locale
                %% instead of `english`, you can typeset the
                %% presentation in either Czech or Slovak,
                %% respectively.
]{babel}        %% typeset as follows:
%%
%%   \begin{otherlanguage}{czech}   ... \end{otherlanguage}
%%   \begin{otherlanguage}{slovak}  ... \end{otherlanguage}
%%
%% These macros specify information about the presentation
% TODO: The date
\title{Examination Timetabling Problem} %% that will be typeset on the
\subtitle{Optimization Methods and Algorithms\\Group 9} %% title page.
\author{s252894 - Piero Macaluso\\s246422 - Ludovico Pavesi\\s254036 - Alberto Romano\\s217189 - Lorenzo Manicone\\s246422 - Donato Tortoriello}
%% These additional packages are used within the document:
\usepackage{ragged2e}  % `\justifying` text
\usepackage{booktabs}  % Tables
\usepackage{tabularx}
\usepackage{tikz}      % Diagrams
\usetikzlibrary{calc, shapes, backgrounds}
\usepackage{amsmath, amssymb}
\usepackage{url}       % `\url`s
\usepackage{listings}  % Code listings
\frenchspacing
\begin{document}
  \frame{\maketitle
 		\note{\textbf{Piero}: Hi, we are the Group 9 of the course Optimization Methods and Algorthms. This group is composed by me, Piero Macaluso, Ludovico Pavesi, Alberto Romano, Lorenzo Manicone and Donato Tortoriello.}}

  \AtBeginSection[]{% Print an outline at the beginning of sections
    \begin{frame}<beamer>
      \frametitle{Outline}
      \tableofcontents[currentsection]
    \end{frame}}
	
	\section{Introduction}
	\begin{frame}
	\note{\textbf{Ludovico}: In our code we decided to use Simulated Annealing Metaheuristic with an implementation of Local Search and Timeslot swap. We managed to implement our code using thread in order to give to create n different path of research at the same time. The only thing that is in common and represents the critical part is the current best. Only one thread can access to the best cost at the same time.}
	\frametitle{\thesection \ \insertsection}
	Our algorithm is based on:
	\begin{itemize}
		\item Simulated annealing with mutations
		\item Local search
		\item Swapping time slot
		\item Multistart
	\end{itemize}
	\end{frame}
	
   \section{Initial solution generation}
   
   \subsection{Scheduling Exams}

	\begin{frame}.
	  \setbeamertemplate{footline}{} 
	   \frametitle{\thesection.\thesubsection \ \insertsubsection}
	ciao
	\end{frame}

   \subsection{Local Search}

	\begin{frame}
	   \frametitle{\thesection.\thesubsection \ \insertsubsection}
	ciao
	\end{frame}

\section{Simulated Mutant Annealing}

\subsection{Temperature and Probability Calibration}

\begin{frame}
\frametitle{\thesection.\thesubsection \ \insertsubsection}
ciao
\end{frame}

\subsection{Neighbor Generation/Mutation}

\begin{frame}
\frametitle{\thesection.\thesubsection \ \insertsubsection}
ciao
\end{frame}

\subsection{Time Slot Swap}

\begin{frame}
\frametitle{\thesection.\thesubsection \ \insertsubsection}
ciao
\end{frame}

\section{Conclusions}

\subsection{Statistics}

\begin{frame}
\frametitle{\thesection.\thesubsection \ \insertsubsection}
ciao
\end{frame}

\subsection{The end}

\begin{frame}
Thanks for listening.
\vfill
Our code is avalable on Github: \href{https://github.com/pimack/OMA9}{github.com/pimack/OMA9}
\end{frame}

\end{document}
