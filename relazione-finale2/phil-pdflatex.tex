%%%%%%%%%%%%%%%%%%%%%%%%%%%%%%%%%%%%%%%%%%%%%%%%%%%%%%%%%%%%%%%%%%%%
%% I, the copyright holder of this work, release this work into the
%% public domain. This applies worldwide. In some countries this may
%% not be legally possible; if so: I grant anyone the right to use
%% this work for any purpose, without any conditions, unless such
%% conditions are required by law.
%%%%%%%%%%%%%%%%%%%%%%%%%%%%%%%%%%%%%%%%%%%%%%%%%%%%%%%%%%%%%%%%%%%%

%\documentclass[notes]{beamer}       % print frame + notes
%\documentclass[notes=only]{beamer}   % only notes
\documentclass{beamer} 
\usetheme[faculty=phil]{fibeamer}
\usepackage[utf8]{inputenc}
\usepackage[
  main=english, %% By using `czech` or `slovak` as the main locale
                %% instead of `english`, you can typeset the
                %% presentation in either Czech or Slovak,
                %% respectively.
]{babel}        %% typeset as follows:
%%
%%   \begin{otherlanguage}{czech}   ... \end{otherlanguage}
%%   \begin{otherlanguage}{slovak}  ... \end{otherlanguage}
%%
%% These macros specify information about the presentation
% TODO: The date
\title{Examination Timetabling Problem} %% that will be typeset on the
\subtitle{Optimization Methods and Algorithms\\Group 9} %% title page.
\author{s252894 - Piero Macaluso\\s246422 - Ludovico Pavesi\\s254036 - Alberto Romano\\s217189 - Lorenzo Manicone\\s246422 - Donato Tortoriello}
%% These additional packages are used within the document:
\usepackage[]{algorithm2e}

\usepackage{ragged2e}  % `\justifying` text
\usepackage{booktabs}  % Tables
\usepackage{tabularx}
\usepackage{tikz}      % Diagrams
\usetikzlibrary{calc, shapes, backgrounds}
\usepackage{amsmath, amssymb}
\usepackage{url}       % `\url`s
\usepackage{listings}  % Code listings
\frenchspacing
\hypersetup{pdfpagemode=FullScreen}
\begin{document}
  \frame{\maketitle
 		\note{\textbf{Piero}: Hi, we are the Group 9 of the course Optimization Methods and Algorthms. This group is composed by me, Piero Macaluso, Ludovico Pavesi, Alberto Romano, Lorenzo Manicone and Donato Tortoriello.}}
	\begin{frame}<beamer>
	\frametitle{Outline}
	\tableofcontents
\end{frame}
  \AtBeginSection[]{% Print an outline at the beginning of sections
    \begin{frame}<beamer>
      \frametitle{Outline}
      \tableofcontents[currentsection]
    \end{frame}}
	
	\section{Introduction}
	\begin{frame}
	\note{\textbf{Ludovico}: In our code we decided to use Simulated Annealing Metaheuristic with an implementation of Local Search and Timeslot swap. We managed to implement our code using thread in order to give to create n different path of research at the same time. The only thing that is in common and represents the critical part is the current best. Only one thread can access to the best cost at the same time.}
	\frametitle{\thesection \ \insertsection}
	Our algorithm is based on:
	\begin{itemize}
		\item Simulated annealing with mutations
		\item Local search
		\item Swapping time slot
		\item Multistart
	\end{itemize}
	\end{frame}
	
   \section{Initial solution generation}
   
   \subsection{Scheduling Exams}

	\begin{frame}
	   \frametitle{\thesection.\thesubsection \ \insertsubsection}
	   \framesubtitle{Ordering Exams}
	   
	\textbf{Goal:} give a useful \alert{order} to the whole set of exams:
	\begin{block}{Number of Unavailable Time Slots}
		This is the number of time slots where there are exams in conflict with the current one.
	\end{block}
	\begin{block}{Number of Conflicts}
		This is the (constant) number of conflicting exams with the current one.
	\end{block}
	\end{frame}

\begin{frame}
\frametitle{\thesection.\thesubsection \ \insertsubsection}
\framesubtitle{The First Critical Point}
\begin{alertblock}{No More Feasible Slots}
	In more complex instances our program gets stuck because the number of unavailable slots of the current exam reaches the maximum.
\end{alertblock}
\pause
\begin{block}{Solution}
	 If the program goes in the tipping point, the software tries to find a way to escape \alert{unscheduling all conflicting exams} and continuing the loop till no exams to be scheduled remain.
\end{block}

\end{frame}

\begin{frame}
\frametitle{\thesection.\thesubsection \ \insertsubsection}
\framesubtitle{The Second Critical Point}
\begin{alertblock}{Instable situations}
	In more complex instances, sometimes the program gets in an endless procedure: it tries to unschedule conflicting exams \alert{without finding the way to conclude} the creation process.

\end{alertblock}
\pause
\begin{block}{Solution}
	\alert{Fixing a maximum number of retries}.
	
	If the software reaches this limit, the creation process restart from the initial point.
\end{block}

\end{frame}

	\begin{frame}
	\frametitle{\thesection.\thesubsection \ \insertsubsection}
	\framesubtitle{The Pseudo-Code}
	\begin{columns}[onlytextwidth]
	\column{.65\textwidth}
	\scalebox{.6}{
	\begin{algorithm}[H]
		%		$backup \gets \text{current solution (even an empty one)}$\;
		%		$sol \gets \text{current solution (even an empty one)}$\;
		$retries \gets 0$\;
		$list \gets \text{exams to be scheduled sorted}$\;
		\While{$list$ is not empty}{
			\If{\textit{retries} > limit}{
				\Return{\alert{no feasible solution found}}\;
				%				$sol \gets backup$\;
				%				restart the alogithm\;
			}
			$E \gets \textit{ first element from }list$\;
			$T \gets \textit{ random available time slot}$\;
			\eIf{$T$ is valid (not null)}{
				schedule $E$ in $T$\;
				remove $E$ from $list$\;
			}{
				\ForEach{conflicting exam $C$ of $E$}{
					unschedule $C$\;
				}
				$retries \gets retries + 1$\;
			}
			sort $list$ again\;
		}
		\Return{\alert{feasible solution found}}\;
	\end{algorithm}}
	\column{.35\textwidth}
\begin{block}{While Loop}
	This code is in a while loop that continues to loop till the code returns \textbf{true}.
\end{block}
\end{columns}
	\end{frame}

	

   \subsection{Local Search}
   
   \begin{frame}
   \frametitle{\thesection.\thesubsection \ \insertsubsection}
   \framesubtitle{Characteristics of Local Search}
   	
   \begin{itemize}
   	\item Best Improvement
   	\item Generation of Neighborhood: O($e$)
   	\item Solution Evaluation: O($e$)
   	\item Real case less than O($e^2$), on average
   	\item Repeates while improvement is above a minimum
   \end{itemize}
 
\end{frame}

	\begin{frame}
	   \frametitle{\thesection.\thesubsection \ \insertsubsection}
	   \framesubtitle{The Pseudo-Code}
	   	\begin{columns}[onlytextwidth]
	   		\column{1\textwidth}
	   \scalebox{.7}{
	\begin{algorithm}[H]
		%		$sol \gets \text{current solution (even an empty one)}$\;
		$list \gets \text{all exams to be scheduled, sorted by }\textit{number of conflicts}$\;
		\ForEach{exam $E$ in $list$}{
			$T \gets \text{time slot where }E\text{ is scheduled}$\;
			$C \gets \text{cost of }E\text{ in }T\text{ according to objective function}$\;
			\ForEach{time slot $T'\neq T$}{
				$C' \gets \text{cost of }E\text{ in }T'\text{ according to objective function}$\;
				\If{$C'<C$}{
					schedule $E$ in $T'$\;
					$C \gets C'$\;
				}
			}
		}
	\end{algorithm}}
\end{columns}
	\end{frame}

\begin{frame}
\frametitle{\thesection.\thesubsection \ \insertsubsection}
\framesubtitle{Results}

\begin{itemize}
	\item Provides the initial delta for Simulated Annealing
	\item Minimum improvement of 10\% works well with given instances
\end{itemize}

\end{frame}

\section{Simulated Mutant Annealing}

\subsection{Temperature and Probability Calibration}

\begin{frame}
\frametitle{\thesection.\thesubsection \ \insertsubsection}
ciao
\end{frame}

\subsection{Neighbor Generation/Mutation}

\begin{frame}
\frametitle{\thesection.\thesubsection \ \insertsubsection}
ciao
\end{frame}

\subsection{Time Slot Swap}

\begin{frame}
\frametitle{\thesection.\thesubsection \ \insertsubsection}
ciao
\end{frame}

\section{Conclusions}

\subsection{Statistics}

\begin{frame}
\frametitle{\thesection.\thesubsection \ \insertsubsection}
ciao
\end{frame}

\subsection{The end}

\begin{frame}
Thanks for listening.
\vfill
Our code is avalable on Github: \href{https://github.com/pimack/OMA9}{github.com/pimack/OMA9}
\end{frame}

\end{document}
po