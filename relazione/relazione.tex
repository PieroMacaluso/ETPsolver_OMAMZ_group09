\documentclass[11pt, a4paper, leqno]{article}
\usepackage[italian]{babel}
\usepackage[utf8]{inputenc}
\usepackage[T1]{fontenc}
\usepackage{parskip}
\usepackage{libertine} % font bello
\usepackage[paper=a4paper,top=1in,bottom=1.1in,right=1in,left=1in]{geometry} % margini

\usepackage{enumitem} % liste più compatte (se ce ne saranno)
\setlist[enumerate]{itemsep=0.0em}
\setlist[itemize]{itemsep=0.0em}
\usepackage[compact]{titlesec}
\usepackage{amsmath}
\usepackage{amssymb}

\newcommand{\nome}[2]{
\begin{minipage}[t]{0.185\linewidth}
	\centering #1\par
	\centering\small (#2)\par
	%\vspace{2em}
	%\rule{\textwidth}{0.6pt} % riga per la firma (nelle relazioni di elettronica serviva)
\end{minipage}
}

% Questo era utile per evidenziare le cose "dubbie e da correggere"
% Poi si elimina il comando cosi' non restano \boh{} in giro
%\usepackage{xcolor}
%\newcommand{\boh}[1]{\textcolor{red}{#1}}

\begin{document}	
	\begin{center}
		{\huge\textbf{OMA Assignment}}\par
		\vspace{0.3em}
		{\large\textbf{Group 9}}\par
		\vspace{1em}
		\nome{Piero Macaluso}{s252894}
		\nome{Lorenzo Manicone}{s217189}
		\nome{Donato Tortoriello}{s205639}
		\nome{Ludovico Pavesi}{s246422}
		\nome{Alberto Romano}{s164371}
	\end{center}
	
	\section{Notation}
	
	\subsection{Data}
	
	\paragraph{$n_{e,e'}$} Number of conflicting students between exam $e$ and $e'$
	
	\paragraph{$\mid S \mid$} Total number of students
	
	\subsection{Variables}
	
	\paragraph{$cal_{t,e}$} ``Calendar'': boolean that is 1 when exam $e$ is placed in timeslot $t$, and 0 otherwise
	
	\paragraph{$y_{e,e',k}$} Boolean that is 1 when exams $e$ and $e'$ are $k$ timeslots apart, 0 otherwise $k \in \{1,\dots,5\}$
	
	\paragraph{$\tau_{e,e'}$} Boolean indicating when there's a conflict between exam $e$ and $e'$
	
	\paragraph{$B_{e,e'}$} Boolean indicating when distance between exam $e$ and $e'$ is positive or negative
	
	\paragraph{$i_{e,e'}$} Distance between two exams between exam $e$ and $e'$
	
	\paragraph{$w_{e,e',k}$} Boolean indicating that distance between exam $e$ and $e'$ is greater than or equal to $k \in \{1,\dots,5\}$ % TODO: prevenire line break
	
	\paragraph{$z_{e,e',k}$} Boolean indicating that distance between exam $e$ and $e'$ is less than or equal to $k \in \{1,\dots,5\}$
	
	\section{LP formulation}
	
	\subsection{Objective function}
	
	\[
		\min\frac{1}{\mid S\mid}\sum_{e,e'} n_{e,e'}\left(16y_{e,e',1}+8y_{e,e',2}+4y_{e,e',3}+2y_{e,e',4}+y_{e,e',5}\right)
	\]
	
	\subsection{Constraints}
	
	\subsubsection{Apply exam}
	
	\[
		\sum_{t} cal_{t,e} = 1 \qquad\forall e
	\]
	
	This is necessary to be sure that every exam is scheduled to exactly one time slot.
	
	\subsubsection{Conflict avoid}
	
	\begin{align*}
		n_{e,e'} \leq M_1\tau_{e,e'} &\qquad \forall e,e'<e \\
		n_{e,e'} \geq \tau_{e,e'} &\qquad \forall e,e'<e
	\end{align*}
	
	$M_1$ is an upper bound, for instance $2\cdot\mid S\mid$. This constraint make sure $\tau$ is set correctly.

	\[
		cal_{t,e}+cal_{t,e'} \leq 2 - \tau \qquad \forall t,e,e'<e
	\]
	
	Two ``conflicting'' exams cannot be placed in the same time slot.
	
	\subsubsection{Distance}
	
	In order to define $i$, this constraint must be linearized:
	
	\[
	i_{e,e'} = \left|\sum_{t} t\cdot cal_{t,e} - \sum_{t} t\cdot cal_{t,e'}\right|
	\]
	
	Defining:
	
	\begin{align*}
		a_e :=& \sum_{t} t\cdot cal_{t,e} \\
		b_{e'} :=& \sum_{t'} t'\cdot cal_{t',e'}
	\end{align*}
	
	after linearization, these constraints are obtained:
	
	\[
	\begin{cases}
		i_{e,e'} \geq a_e-b_{e'}&\quad \forall e,e'<e\\
		i_{e,e'} \geq b_{e'}-a_e&\quad \forall e,e'<e
	\end{cases}
	\]
	
	\[
	\begin{cases}
	a_e-b_{e'} + MB \geq i_{e,e'}&\quad \forall e,e'<e\\
	b_{e'}-a_e + M(1-B) \geq i_{e,e'}&\quad \forall e,e'<e\\
	a_e-b_{e'} \leq M(1-B)&\quad \forall e,e<e'\\
	b_{e'}-a_e \leq MB&\quad \forall e,e'<e
	\end{cases}
	\]
	
	\subsubsection{Penality function}
	
	\begin{align*}
		i_{e,e'} \geq k w_{e,e',k} &\qquad\forall k \in \{1,\dots,5\},e,e'<e\\
		i_{e,e'} \leq (k-1) + M  w_{e,e',k} &\qquad\forall k \in \{1,\dots,5\},e,e'<e
	\end{align*}
	
	This is in order to apply values to $w$.
	
	\begin{align*}
		i_{e,e'} \geq \left(k+1\right)\left(1-z_{e,e',k}\right) &\qquad\forall k \in \{1,\dots,5\},e,e'<e\\
		i_{e,e'} \leq k + M  (1-z_{e,e',k}) &\qquad\forall k \in \{1,\dots,5\},e,e'<e
	\end{align*}
	
	This is in order to apply values to $z$.
	
	\begin{align*}
		w_{e,e',k} + z_{e,e',k} \leq y_{e,e',k} + 1 &\qquad\forall k \in \{1,\dots,5\},e,e'<e\\
		w_{e,e',k} + z_{e,e',k} \geq 2y_{e,e',k} &\qquad\forall k \in \{1,\dots,5\},e,e'<e
	\end{align*}
	
	This is in order to apply values to $y$, which is needed in the objective function, using $w$ and $z$.
	
	\subsubsection{Domain}
	
	\[
		i_{e,e'} \in \mathbb{Z}^{+}
	\]
	
	\[
		cal_{t,e},\ y_{e,e',k},\ \tau_{e,e'},\ B_{e,e'},\ w_{e,e',k},\ z_{e,e',k} \in \{0,1\}
	\]
\end{document}
